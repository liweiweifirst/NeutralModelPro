\documentclass[11pt]{article}
\include{noteinclude}

\begin{document}
Last updated: \today

\tableofcontents

\section{produce\_results.pro}

This procedure is the starting point for creating results from model outputs. 
An input structure or file and a format structure or file are required. The 
modeloutput files associated with the input file should be created before 
running this routine. 

The possible result types are: 
\begin{itemize}
\item image = 2-D image from a specfied viewing geometry
\item voronoi image = 2-D image using the voronoi regions method
\item los = the line-of-sight results along specific spacecraft trajectory and 
look direction
\item points = the density sampled at specified points or along a spacecraft 
trajectory
\end{itemize}

The quantities that can be determined are:
\begin{itemize}
\item Density
\item Column density
\item Intensity (multiple excitation processes possible)
\item Spectrum (simulated spectrum over wavelength range) 
\end{itemize}

The rest of this section outlines the produce\_results procedure.

\begin{enumerate}
\item Function call: \texttt{produce\_results, inputtemp, formattemp, 
  data=data, npackets=npackets, savefile=savefile}
  \begin{enumerate}
  \item \texttt{inputtemp} can be:
    \begin{enumerate}
    \item an input file that can be restored
    \item an input structure
    \item list of output files to restore 
      \begin{itemize}
      \item The input structure is determined from the first output file in 
	the list
      \item If the input structure is not the same for each output file, the 
	results will be unpredictable (i.e., wrong).
      \end{itemize}
    \end{enumerate}
  \item \texttt{formattemp} can be:
    \begin{enumerate}
    \item a format file that can be restored
    \item a \hyperref[format]{format structure}
    \end{enumerate}
  \item \texttt{data} = points at which to produce results
    \begin{enumerate}
    \item if producing LOS results, data is a structure with (x, y, z, xbore, 
      ybore, zbore)
    \item if producing density or column density, data is a structure with 
      (x, y, z)
    \end{enumerate}
  \item \texttt{npackets} = minimum number of packets needed to produce a 
    result (default = 0)
  \item \texttt{savefile} = file to save the loaded packets into - still 
    working on this option
  \end{enumerate}

\item Common block \textit{results} contains: 
  \begin{enumerate} 
  \item input = the input structure for this model run
  \item format = the \hyperref[format]{format structure} for this result
  \item SystemConsts
  \item stuff
  \item gvalue
  \item plasma = current plasma parameters for the simulation
  \end{enumerate}
\item \textit{stuff} structure contains:
  \begin{enumerate}
  \item aplanet = distance of planet from the sun (AU)
  \item vr = radial velocity of planet relative to the sun (km/s)
  \item atoms\_per\_packet = number of atoms 1.0 packets represents
  \item mod\_rate = rate packets are ejected from the surface
  \item totalsource = total number of packets ejected
  \end{enumerate}

\item Restore the inputs and determine which output files to use 

\item Restore the SystemConsts structure and determine the distance and radial 
  velocity of the central planet based on input.geometry.taa \\
  \textbullet\ function calls to {\color{blue}SystemConstants} and 
  {\color{blue}planet\_dist}

\item Determine \textit{totalpackets} = total number of packets available
  \begin{enumerate} 
  \item extracts the number of saved packets from the output file headers and
  \item If the \textit{npackets} keyword is not set, then $npackets=0$.
  \item If no output files are found, then $totalpackets = 0$
  \item If $totalpackets < npackets$ then there is nothing to do. 
  \end{enumerate}

\item Restore the format structure if the filename is given as an input

\item Determine \textit{stuff.totalsource} by extracting the values from the 
  output files 
  \begin{enumerate}
  \item $totalsource = \sum output.frac0$ \rarrow\ total starting fractional 
    values for the packets
  \item $totalpackets$ is the number of packets ignoring the initial fractional 
    value
  \end{enumerate}

\item $stuff.mod\_rate = stuff.totalsource / input.options.endtime$ \rarrow\ 
  packets ejected per second

\item $stuff.atoms\_per\_packet = format.strength \times 10^{26}$ \rarrow\ 
  number of atoms each packet represents. $fomat.strength$ = source rate in 
  units of $10^{26}$ atoms/sec.

\item if $format.quantity$ = `intensity', then setup the intensity parameters \\
  \textbullet\ run procedure 
  \hyperref[intensitysetup]{results\_intensity\_setup}

\item Run the appropriate results production program based on format.type:
  \begin{enumerate}
  \item `image': Create an image from a specified viewing geometry \\
    \textbullet\ result = 
    \hyperref[image]{produce\_image(files, savefile=savefile)}
  \item `voronoi image': Create an image from a specified viewing geometry 
  using the voronoi region method (still in progress) \\
    \textbullet\ result = 
    \hyperref[voronoiimage]{produce\_voronoi\_image(files, savefile=savefile)}
  \item `los': Determine result along a trajectory line of sight \\
    \textbullet\ result = \hyperref[los]{produce\_los(files, savefile=savefile)} 
  \item `points': Determine density at specified points \\ 
    \textbullet\ result = 
    \hyperref[points]{produce\_density(files, savefile=savefile)}
  \end{enumerate}
\end{enumerate}

\section{Format Structures \label{format}}

\section{results\_intensity\_setup \label{intensitysetup}}
\begin{enumerate}
\item procedure is in the file \textit{results\_functions\_4.0.pro}
\item This routine loads the gvalues and plamsa info for resonant scattering
and electron impact excitation
\item Resonant scattering emission:
  \begin{enumerate}
  \item format.emission.mechanism must contain 'resscat'
  \item g-value determined from {\color{blue}get\_gvalue}, which gives the
  gvalues for specified atom at specified distance from the sun as function of
  radial velocity. The g-value structure contains:
    \begin{enumerate}
    \item species: the emitting atom
    \item a: distance from sun (AU)
    \item wavelength: array of emission line wavelengths (\AA)
    \item v: array of radial velocities relative to the sun (km s$^{-1}$)
    \item g: ($n_v \times n_\lambda$)-array with g-value as function of $v_r$ 
    and $\lambda$ (photons atom$^{-1}$ s$^{-1}$)
    \item radaccel: radiation accelleration as function of $v_r$ (cm s$^{-2}$)
    \end{enumerate}
  \end{enumerate}
\item Electron impact excitation
  \begin{enumerate}
  \item format.emission.mechanism must contain 'eimp'
  \item \textbf{This has not been set up correctly yet.}
  \end{enumerate}
\end{enumerate}

\section{produce\_image \label{image}}

This procedure produces a 2-D image from a specified viewing geometry
\rarrow\ See Section~\ref{format} for required parameters. The user does
not call this function directly -- it is called from
{\color{blue}produce\_results}.

The procedure is outlined below:
\begin{enumerate}
\item Determine the image origin -- There are some differences in the
calculations depending on whether the image center is the planet or a
satellite.

\item Image dimensions:
  \begin{enumerate}
  \item Size of the output image (number of pixels) is given by
  \texttt{format.geometry.dims}
  \item The image center is given by \texttt{format.geometry.center} measured  
  in $R_{obj}$ relative to the center of the image origin
  (\texttt{format.geometry.origin}). 
  \item The width of the image is given by \texttt{format.geometry.width}
  measured in $R_{obj}$.
  \item Note that \texttt{geometry.dims}, \texttt{geometry.width}, and
  \texttt{geometry.center} are all 2-element vectors.
  \end{enumerate}

\item Coordinate system
  \begin{enumerate}
  \item The packets are rotated into a reference frame with the x-axis aligned
  along the image horizontal axis, the z-axis aligned along the image vertical
  axis, and the y-axis along the image line of sight with the observer located
  at $(0,-\infty,0)$
  \item Image scale: 
  \begin{equation}
  scale = \frac{geometry.width}{geometry.dims-1}
  \end{equation}
  \end{enumerate}

\end{enumerate}

\section{produce\_voronoi\_image \label{voronoiimage}}

\section{produce\_los \label{los}}

\section{produce\_points \label{points}}

\end{document}
